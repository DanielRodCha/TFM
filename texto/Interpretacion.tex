\chapter{Interpretación algebraica de la lógica}\label{sec:interp}

En este capítulo se estudiarán las principales relaciones entre la lógica proposicional y los 
polinomios con coeficientes en cuerpos finitos, centrando la atención en $\mathbb{F}_2$, 
el cuerpo finito con dos elementos.\\

La idea principal que subyace en la interpretación algebraica de la lógica es la de hacerle corresponder a cada fórmula un polinomio de forma que la función valor de verdad inducida por la fórmula se pueda entender como una función polinomial de $\mathbb{F}_2$. En otras palabras, se persigue que si la fórmula es verdadera, el valor del polinomio que tiene asociado es 1; mientras que si la fórmula es falsa, el polinomio vale 0.\\

En la Figura \ref{fig:esquema} (abajo) se muestra una representación gráfica de la relación entre las fórmulas proposicionales y los polinomios de $\mathbb{F}_2[x]$. Destacar que se usa el ideal $\mathbb{I}_2 :=(x_1+x_1^2,...,x_n+x_n^2)\subseteq\mathbb{F}_2[x]$ y que $proj$ es la proyección natural sobre el anillo cociente.

\vspace{0.5cm}
\begin{figure}[h]
	\centering
		\includegraphics[scale=0.46]{imagenes/conmu.png}
	\caption{Relación entre las fórmulas proposicionales y $\mathbb{F}[x]_2$}
	\label{fig:esquema}
\end{figure}
\vspace{0.5cm}

Destacar que se paralelizará la exposición de la teoría y el desarrollo de las implementaciones en Haskell. Teniendo en cuenta que el objetivo principal de dichos programas es la obtención de una herramienta eficiente para resolver el problema SAT, tema que se tratará con más detalle en los próximos capítulos.
 
\entrada{Logica}
\entrada{Haskell4Maths}
\entrada{F2}
\entrada{Transformaciones}