\documentclass[10pt,a4paper]{beamer}
\usepackage[spanish,activeacute]{babel}
\usepackage[utf8]{inputenc}
%\usepackage[right=2cm,left=3cm]
\usepackage{mathrsfs}
\usepackage{mathtools}
\usepackage{amsthm}
\usepackage{graphicx}
\usepackage{amsfonts}
\usepackage{amsmath}
\usepackage{amssymb, setspace}

\usepackage{parskip}
\setlength{\parindent}{30pt}
\usepackage{ragged2e}

\newtheorem{definicion}{Definición:}[section]



% Esto es para poder escribir acentos directamente:
\usepackage[utf8]{inputenc}
% Esto es para que el LaTeX sepa que el texto está en español:
\usepackage[spanish]{babel}

% Paquetes de la AMS:
\usepackage{amsmath, amsthm, amsfonts,amssymb, wasysym}
\usepackage{synttree} % Para hacer arboles
\usepackage{graphicx} % Imagenes
\usepackage{cancel}
%\usepackage[usenames]{color} %colores
\usepackage{multirow} % para las tablas
\usepackage{wrapfig} % imagenes al lado de texto
\usepackage{float} % colocar imagenes donde se quiera


%\usepackage{a4wide}
\usepackage{color}
\usepackage{mathpazo} 
\usepackage[scaled=.90]{helvet}
\usepackage{cmtt}
\renewcommand{\ttdefault}{cmtt}

\newcommand{\mathsym}[1]{{}}
\newcommand{\unicode}[1]{{}}

\theoremstyle{definition}
\newtheorem{thm}{Teorema}
\newtheorem{cor}[thm]{Corolario}
\newtheorem{lem}[thm]{Lema}
\newtheorem{prop}[thm]{Proposición}
\newtheorem{defn}[thm]{Definición}
\theoremstyle{remark}
\newtheorem{pru}[thm]{Prueba}
\newtheorem{rem}[thm]{Observación}

\def\RR{\mathbb{R}}
\def\ZZ{\mathbb{Z}}
\def\NN{\mathbb{N}}
\def\PP{\mathbb{P}}
\def\NP{\mathbb{NP}}
\def\!!!{ \rightarrow \leftarrow }

\def\MM{\textbf{M}}
\newcommand{\QED}{\hfill {\qed}}

\usepackage{fancyvrb} 


\usetheme{CambridgeUS}
\usecolortheme{whale}
\useoutertheme{sidebar}
\useinnertheme{circles}
\setbeamercolor{frametitle}{fg=white}


\title[Modelo del ecosistema de una población de Pandas Gigantes del GPBB]{Modelo del ecosistema de una población de Pandas Gigantes de la \textit{Chengdu Research Base of Giant Panda Breeding}}
\author{Daniel R. Chavarría}
\institute[US]{Simulación y Análisis Computacional en Biología de Sistemas \\Departamento de Ciencias de la Computación e Inteligencia Artificial \\ Universidad de Sevilla} 
\date{6 de Marzo de 2017}
\setbeamertemplate{navigation symbols}{}

\AtBeginSection[]
{\begin{frame}
\frametitle{Índice}
\begin{minipage}{\textwidth}
\linespread{1.4}
\tableofcontents[
	currentsection]
\end{minipage}
\end{frame}}

\begin{document}
\frame{\titlepage}


%--------------------------------------------------------------------------
\begin{frame}
\frametitle{Interrupid ante cualquier duda}
\includegraphics[scale=0.4]{bienvenidos}
\end{frame}

\begin{frame}
\frametitle{El protagonista:}
\includegraphics[scale=0.5]{panda2}
\end{frame}

\begin{frame}
\frametitle{Índice}
\begin{minipage}{\textwidth}
\linespread{1.4}
\tableofcontents
\end{minipage}
\end{frame}

\section{Panda Gigante}

\begin{frame}
\frametitle{Panda Gigante}
\framesubtitle{Información general}
\begin{itemize}
\item Mamífero de la orden de los carnívoros
\item Familia de ....
\item
\item
%\item Familia de los Úrsidos
%\item Habita en regiones de China Central, a unos 3500 msnm.
%\item Actualmente hay unos 2000 en libertad y 200 en cautividad.
\end{itemize}
\end{frame}

\begin{frame}
\frametitle{¿Mapaches u osos?}
\framesubtitle{}
\includegraphics[scale=2.3]{pandarojo} \hspace{1cm}
\includegraphics[scale=0.16]{anteojos}

\hspace{1cm} \textbf{Panda rojo} \hspace{1.3cm} \textbf{VS} \hspace{0.8cm} \textbf{Oso de Anteojos}
\end{frame}

\begin{frame}
\frametitle{Panda Gigante}
\framesubtitle{Información general}
\begin{itemize}
\item Mamífero de la orden de los carnívoros
%\item Familia de ....
\item Familia de los Úrsidos
\item Habita en regiones de China Central, a unos 3500 msnm.
\item Actualmente hay unos 2000 en libertad y 200 en cautividad.
\item Debido al clima, el panda ha perdido el hábito de la hibernación.
\end{itemize}
\end{frame}

\begin{frame}
\frametitle{Panda Gigante}
\framesubtitle{Anatomía}
\begin{itemize}
\item En la edad adulta miden unos 0.75 \textit{m} de altura (sobre las 4 patas) y 1.5 \textit{m} de longitud.
\item De adultos, pueden pesar entre 70 y 125 kg.
\item Las crías al nacer pesan entre 90 y 130 gramos y sin apenas pelo.
\end{itemize}
\end{frame}

\begin{frame}
\frametitle{Panda Gigante}
\framesubtitle{Anatomía}
\includegraphics[scale=0.45]{bebe}
\end{frame}

\begin{frame}
\frametitle{Panda Gigante}
\framesubtitle{Anatomía}
\begin{itemize}
\item En la edad adulta miden unos 0.75 \textit{m} de altura (sobre las 4 patas) y 1.5 \textit{m} de longitud.
\item De adultos, pueden pesar entre 70 y 125 kg.
\item Las crías al nacer pesan entre 90 y 130 gramos y sin apenas pelo.
\item La pata tiene 5 dedos y un "pulgar" (hueso sesamoideo).
\item A diferencia del resto de osos, sus pupilas son como las de los gatos.
\item Existen dos clases de pelaje.
\end{itemize}
\end{frame}


\begin{frame}
\frametitle{Panda Gigante}
\framesubtitle{Panda de Sichuan }
\includegraphics[scale=1]{panda1}
\end{frame}

\begin{frame}
\frametitle{Panda Gigante}
\framesubtitle{Panda de Quingling}
\begin{center}
\includegraphics[scale=0.5]{quinling}
\end{center}



\end{frame}

\begin{frame}
\frametitle{Panda Gigante}
\framesubtitle{Alimentación}
\begin{itemize}
\item La mayoría de su alimentación consiste en bambú. Su sistema digestivo no está adaptado para digerirlo, por lo que debe ingestar entre 14 y 30 \textit{kg} de bambú diarios. Tarea que le lleva 14 horas al día.
\item En muy raras ocasiones ingiere otro tipo de alimentos (insectos huevos, roedores) como fuente de proteínas.
\end{itemize}
\vspace{0.2cm}
\begin{center}
\includegraphics[scale=0.4]{panda}
\end{center}
\end{frame}

\section{GPBB}
\begin{frame}
\frametitle{\large{Chengdu Research Base of Giant Panda Breeding}}
\framesubtitle{}
\begin{itemize}
\item Fundada en 1987.
\item Es la encargada de la investigación, alimentación y conservación del Panda Gigante alrededor de todo el mundo.
\end{itemize}
\includegraphics[scale=0.3]{gpbb2} \hspace{0.5cm}
\includegraphics[scale=0.3]{gpbb1}

\includegraphics[scale=0.15]{gpbb3} \hspace{0.51cm}
\includegraphics[scale=0.11]{gpbb4}
\end{frame}

\begin{frame}
\frametitle{\large{Chengdu Research Base of Giant Panda Breeding}}
\framesubtitle{El Panda Gigante en los Zoológicos}
\begin{itemize}
\item Es el animal más caro de mantener. El costo de mantener un panda es cinco veces mayor que el del elefante, el segundo mayor en cuanto a costo.
\item Los zoológicos deben pagar al gobierno chino la suma de 2 millones de dólares por año en concepto de derechos y honorarios.
\item Zoológicos de fuera de China que tienen (por continentes):
\begin{itemize}
\item[--] América: 7
\item[--] Asia: 6
\item[--] Australia: 1
\item[--] Europa: 6
\end{itemize}
\end{itemize}

\end{frame}

\begin{frame}
\frametitle{\large{Chengdu Research Base of Giant Panda Breeding}}
\framesubtitle{El Panda Gigante en los Zoológicos}
\begin{center}
\includegraphics[scale=0.8]{zoom}

\textbf{¡Tranquilos, Madrid es uno de los afortunados!}
\end{center}
\end{frame}

\section{Descripción del modelo}

\begin{frame}
\frametitle{Descripción del Modelo}
\framesubtitle{Notación}
\begin{small}
Se denotará por un subíndice $i=1$ si los individuos son machos y por $i=2$ si son hembras.\\

\begin{itemize}
\item[•] $k_{i,1}$: edad a la que se alcanza el tamaño \textit{sub-adulto}.
\item[•] $k_{i,2}$: edad a la que se alcanza el tamaño \textit{adulto joven}.
\item[•] $k_{i,3}$: edad a la que se alcanza el tamaño \textit{medio adulto}.
\item[•] $k_{i,4,1}$: edad a la que se alcanza el tamaño \textit{anciano} (entre los 17 y los 26).
\item[•] $k_{i,4,2}$: edad a la que se alcanza el tamaño \textit{anciano} (entre los 27 y los 35).
\item[•] $k_{i,5}$: máxima esperanza de vida en el ecosistema dado.
\item[•] $k_{i,6}$: tasa de mortalidad en pandas gigantes \textit{infantes}.
\item[•] $k_{i,7}$: tasa de mortalidad en pandas gigantes \textit{sub-adultos}.
\item[•] $k_{i,8}$: tasa de mortalidad en pandas gigantes \textit{adultos jóvenes}.
\item[•] $k_{i,9}$: tasa de mortalidad en pandas gigantes \textit{medio adultos}.
\item[•] $k_{i,10}$: tasa de mortalidad en pandas gigantes \textit{ancianos} (17 a 26).
\item[•] $k_{i,11}$: tasa de mortalidad en pandas gigantes \textit{ancianos} (26 a 35).
\end{itemize}
\end{small}
\end{frame}



\begin{frame}
\frametitle{Descripción del Modelo}
\framesubtitle{Notación}
\begin{small}
\begin{itemize}
\item[•] $g_{1}$: número de descendientes machos por año.
\item[•] $g_{2}$: número de descendientes hembra por año.
\item[•] $g_{3}$: reservas anuales de brotes de bambú del GPBB (en kg).
\item[•] $g_{4}$: reservas anuales de bambú del GPBB (en kg).
\item[•] $g_{5}$: reservas anuales de otro tipo de comida (i.e. manzanas, carne, leche) del GPBB (en kg).
\end{itemize}
\end{small}
\end{frame}

\begin{frame}
\frametitle{Descripción del Modelo}
\framesubtitle{Notación}
\begin{small}
\begin{itemize}
\item[•] $f_{i,1}$: cantidad anual necesaria de brotes de bambú (en kg) de acuerdo a las necesidades energéticas de un panda gigante \textit{infante}.
\item[•] $f_{i,2}$: cantidad anual necesaria de bambú (en kg) de acuerdo a las necesidades energéticas de un panda gigante \textit{infante}.
\item[•] $f_{i,3}$: cantidad anual necesaria de otros alimentos (en kg) de acuerdo a las necesidades energéticas de un panda gigante \textit{infante}.
\item[•] $f_{i,4}$: cantidad anual necesaria de brotes de bambú (en kg) de acuerdo a las necesidades energéticas de un panda gigante \textit{sub-adulto}.
\end{itemize}
\end{small}
\end{frame}

\begin{frame}
\frametitle{Descripción del Modelo}
\framesubtitle{Notación}
\begin{small}
\begin{itemize}
\item[•] $f_{i,5}$: cantidad anual necesaria de bambú (en kg) de acuerdo a las necesidades energéticas de un panda gigante \textit{sub-adulto}.
\item[•] $f_{i,6}$: cantidad anual necesaria de otros alimentos (en kg) de acuerdo a las necesidades energéticas de un panda gigante \textit{sub-adulto}.
\item[•] $f_{i,7}$: cantidad anual necesaria de brotes de bambú (en kg) de acuerdo a las necesidades energéticas de un panda gigante \textit{adulto} ó \textit{anciano}.
\item[•] $f_{i,8}$: cantidad anual necesaria de bambú (en kg) de acuerdo a las necesidades energéticas de un panda gigante \textit{adulto} ó \textit{anciano}.
\item[•] $f_{i,9}$: cantidad anual necesaria de otros alimentos (en kg) de acuerdo a las necesidades energéticas de un panda gigante \textit{adulto} ó \textit{anciano}.
\end{itemize}
\end{small}
\end{frame}

\begin{frame}
\frametitle{Descripción del Modelo}
\framesubtitle{Notación}
\begin{small}
\begin{itemize}
\item[•] $cmin$: mínimo número de pandas gigantes rescatados anualmente.
\item[•] $cmax$: máximo número de pandas gigantes rescatados anualmente.
\item[•] $cmaxage$: máxima edad de los pandas gigantes rescatados.
\item[•] $pc_{c}$: probabilidad de rescatar $c$ individuos.
\item[•] $pg_{i}$: probabilidad de que los individuos rescatados sean del género $i$.
\item[•] $pa_{j}$: probabilidad de que los individuos rescatados tengan la edad $j$.
\end{itemize}
\end{small}
\end{frame}



\begin{frame}
\frametitle{Descripción del Modelo}
\framesubtitle{Símbolos del modelo}
\begin{small}
\begin{itemize}
\item[•] $q_{i,j}$: número de pandas gigantes de edad $j$.
\item[•] $X_{i,j}$: individuos de edad $j$ antes del módulo de reproducción.
\item[•] $Y_{i,j}$: individuos de edad $j$ en el módulo de mortalidad.
\item[•] $Z_{i,j}$: supervivientes de edad $j$.
\item[•] $W_{i,j}$: individuos de edad $j$ después de módulo de alimentación.
\item[•] $C_{i,j}$: individuos rescatados de edad $j$.
\end{itemize}
\end{small}
\end{frame}

\begin{frame}
\frametitle{Descripción del Modelo}
\framesubtitle{Símbolos del modelo}
\begin{small}
\begin{itemize}
\item[•] $S$: brotes de bambú.
\item[•] $B$: bambú.
\item[•] $O$: otros alimentos.
\item[•] $F$: objeto auxiliar necesario para que el sistema generen una nueva cantidad de alimento al principio de cada ciclo de tiempo.
\item[•] $N$: objeto auxiliar necesario para que el sistema generen nuevos nacimientos al principio de cada ciclo de tiempo.
\item[•] $A$: objeto auxiliar que dispara los rescates.
\end{itemize}
\end{small}
\end{frame}

\begin{frame}
\frametitle{Descripción del Modelo}
\framesubtitle{Estructura de membranas}
\begin{center}
$\mu=[\,[\;\;]_{2}\,]_{1}$

\includegraphics[scale=0.5]{estructura}
\end{center}

\end{frame}

\begin{frame}
\frametitle{Descripción del Modelo}
\framesubtitle{Multiconjunto inicial}
\begin{center}
$\mathcal{M}_1=\{X^{q_{i,j}}_{i,j} : 1\leq i \leq 2, 1 \leq j \leq k_{i,5}\}$, $\mathcal{M}_2=\{F \;\; N \;\; A\}$

\includegraphics[scale=0.5]{init}
\end{center}
\end{frame}

\begin{frame}
\frametitle{Descripción del Modelo}
\framesubtitle{Esquema de módulos}
\begin{center}
\includegraphics[scale=0.3]{esquemageneral}
\end{center}
\end{frame}

\begin{frame}
\frametitle{Descripción del Modelo}
\framesubtitle{Módulo de Inicialización}
\begin{center}
\begin{small}
\begin{itemize}
\item[•] Generación de los objetos relacionados con la comida:
$$r_1 \equiv [F]^0_2 \,\longrightarrow \,F \, [S^{g_3}\;B^{g_4}\;O^{g_5}]^+_2 $$
\end{itemize}
\end{small}
\includegraphics[scale=0.5]{initr}
\end{center}
\end{frame}

\begin{frame}
\frametitle{Descripción del Modelo}
\framesubtitle{Módulo de Reproducción}
\begin{small}
\begin{itemize}
\item[•] Reglas relacionadas con el nacimiento:
$$r_2 \equiv [N]^0_2 \,\longrightarrow \,N \, [Y^{g_1}_{1,0}\;Y^{g_2}_{2,0}]^+_2$$
\end{itemize}
\end{small}
\begin{center}
\includegraphics[scale=0.5]{nacim}
\end{center}
\end{frame}



\begin{frame}
\frametitle{Descripción del Modelo}
\framesubtitle{Módulo de Reproducción}
\begin{small}
\begin{itemize}
\item[•] Reglas de crecimiento
$$r_3 \equiv [X_{i,j}]^0_2 \, \longrightarrow \,  [Y_{i,j}]^+_2 , \text{ con } \left \{ \begin{matrix} 1 \leq i \leq 2
\\ 1\leq j \leq k_{i,5} \end{matrix}\right. $$
\end{itemize}
\end{small}
\begin{center}
\includegraphics[scale=0.5]{grow}
\end{center}
\end{frame}


\begin{frame}
\frametitle{Descripción del Modelo}
\framesubtitle{Módulo de Rescate}
\begin{small}
\begin{itemize}
\item[•] Probabilidad de rescatar $c$ individuos:
$$r_4 \equiv [A]^0_2 \, \xrightarrow{pc_c} \,  A\; C^c \; [\;]^+_2 , \text{ con }  cmin \leq c \leq cmax $$
\end{itemize}
\end{small}
\begin{center}
\includegraphics[scale=0.5]{resc1}
\end{center}
\end{frame}

\begin{frame}
\frametitle{Descripción del Modelo}
\framesubtitle{Primera ejecución}
\begin{center}
\includegraphics[scale=0.4]{paso1}
\end{center}
\end{frame}


\begin{frame}
\frametitle{Descripción del Modelo}
\framesubtitle{Módulo de Mortalidad}
\begin{small}
\begin{itemize}
\item[•] Pandas Gigantes \textit{infantes} que sobreviven:
$$r_7 \equiv [Y_{i,j} \,\xrightarrow{1-k_{i,6}} \, Z_{i,j} ]^+_2 , \text{ con } \left \{ \begin{matrix} 1 \leq i \leq 2
\\ 0 \leq j < k_{i,1} \end{matrix}\right. $$
\item[•] Pandas Gigantes \textit{infantes} que fallecen:
$$r_8 \equiv [Y_{i,j} \,\xrightarrow{k_{i,6}} \, \lambda ]^+_2 , \text{ con } \left \{ \begin{matrix} 1 \leq i \leq 2
\\ 0 \leq j < k_{i,1} \end{matrix}\right. $$
\end{itemize}
\begin{center}
.\\
.\\
.
\end{center}
\begin{itemize}
\item[•] Pandas Gigantes que alcanzan la máxima esperanza de vida:
$$r_{19} \equiv [Y_{i,k_{i,5}} \,\longrightarrow \, \lambda ]^+_2 , \text{ con } 1 \leq i \leq 2 $$
\end{itemize}
\end{small}
\end{frame}

\begin{frame}
\frametitle{Descripción del Modelo}
\framesubtitle{Módulo de Mortalidad}
\begin{center}
\includegraphics[scale=0.5]{muerte}
\end{center}
\end{frame}

\begin{frame}
\frametitle{Descripción del Modelo}
\framesubtitle{Módulo de Rescate}
\begin{small}
\begin{itemize}
\item[•] Probabilidad de que los individuos rescatados tengan género $i$:
$$r_5 \equiv [C \,\xrightarrow{pg_i} \, C_i ]^0_1 , \text{ con }  1 \leq i \leq 2 $$
\end{itemize}
\end{small}
\begin{center}
\includegraphics[scale=0.5]{resc2}
\end{center}
\end{frame}

\begin{frame}
\frametitle{Descripción del Modelo}
\framesubtitle{Segunda ejecución}
\begin{center}
\includegraphics[scale=0.4]{paso2}
\end{center}
\end{frame}

\begin{frame}
\frametitle{Descripción del Modelo}
\framesubtitle{Módulo de Alimentación}
\begin{small}
\begin{itemize}
\item[•] Proceso de alimentación de los Pandas Gigantes \textit{infantes}:
$$r_{20} \equiv [Z_{i,j} \; S^{f_{i,1}} \; B^{f_{i,2}} \; O^{f_{i,3}}]^+_2 \,\longrightarrow \, [W_{i,j}]^0_2 , \text{ con } \left \{ \begin{matrix} 1 \leq i \leq 2
\\ 0 \leq j < k_{i,1} \end{matrix}\right. $$
\item[•] Proceso de alimentación de los Pandas Gigantes \textit{subadultos}:
$$r_{21} \equiv [Z_{i,j} \; S^{f_{i,4}} \; B^{f_{i,5}} \; O^{f_{i,6}}]^+_2 \,\longrightarrow \, [W_{i,j}]^0_2 , \text{ con } \left \{ \begin{matrix} 1 \leq i \leq 2
\\ < k_{i,1} \leq j < k_{i,2} \end{matrix}\right. $$
\item[•] Proceso de alimentación de los Pandas Gigantes \textit{adultos} y \textit{ancianos}:
$$r_{22} \equiv [Z_{i,j} \; S^{f_{i,7}} \; B^{f_{i,8}} \; O^{f_{i,9}}]^+_2 \,\longrightarrow \, [W_{i,j}]^0_2 , \text{ con } \left \{ \begin{matrix} 1 \leq i \leq 2
\\ < k_{i,2} \leq j < k_{i,5} \end{matrix}\right. $$
\end{itemize}
\end{small}
\end{frame}

\begin{frame}
\frametitle{Descripción del Modelo}
\framesubtitle{Módulo de Alimentación}
\begin{center}
\includegraphics[scale=0.5]{feed}
\end{center}
\end{frame}

\begin{frame}
\frametitle{Descripción del Modelo}
\framesubtitle{Módulo de Rescate}
\begin{small}
\begin{itemize}
\item[•] Probabilidad de que los individuos rescatados tengan la edad $j$:
$$r_6 \equiv [C_i \,\xrightarrow{pg_i} \, C_{i,j+1+ \lfloor \frac{j}{3} \rfloor} ]^0_1 , \text{ con } \left \{ \begin{matrix} 1 \leq i \leq 2
\\ 0 \leq j \leq cmaxage \end{matrix}\right. $$
\end{itemize}
\end{small}
\begin{center}
\includegraphics[scale=0.5]{resc3}
\end{center}
\end{frame}

\begin{frame}
\frametitle{Descripción del Modelo}
\framesubtitle{Tercera ejecución}
\begin{center}
\includegraphics[scale=0.4]{paso3}
\end{center}
\end{frame}

\begin{frame}
\frametitle{Descripción del Modelo}
\framesubtitle{Módulo de Rescate}
\begin{small}
\begin{itemize}
\item[•] Reglas que eliminan la comida sobrante:
$$r_{23} \equiv [S \, \longrightarrow \, \lambda]^0_2$$
$$r_{24} \equiv [B \, \longrightarrow \, \lambda]^0_2$$
$$r_{25} \equiv [O \, \longrightarrow \, \lambda]^0_2$$
\end{itemize}
\end{small}
\begin{center}
\includegraphics[scale=0.5]{act1}
\end{center}
\end{frame}

\begin{frame}
\frametitle{Descripción del Modelo}
\framesubtitle{Módulo de Rescate}
\begin{small}
\begin{itemize}
\item[•] Reglas que preparan el principio del próximo ciclo:
$$r_{26} \equiv [W_{i,j}]^0_2 \, \longrightarrow \,  X_{i,j+1} \, [\;]^0_2 , \text{ con } \left \{ \begin{matrix} 1 \leq i \leq 2
\\ 1\leq j < k_{i,5} \end{matrix}\right. $$
$$r_{27} \equiv C_{i,j} \, [\;]^0_2 \, \longrightarrow \, X_{i,j+1}\, [\;]^0_2 , \text{ con } \left \{ \begin{matrix} 1 \leq i \leq 2
\\ 1\leq j < k_{i,5} \end{matrix}\right. $$
$$r_{28} \equiv F \, [\;]^0_2 \, \longrightarrow \, [F]^0_2$$
$$r_{29} \equiv N \, [\;]^0_2 \, \longrightarrow \, [N]^0_2$$
$$r_{30} \equiv A \, [\;]^0_2 \, \longrightarrow \, [A]^0_2$$
\end{itemize}
\end{small}
\end{frame}

\begin{frame}
\begin{center}
\includegraphics[scale=0.5]{act2}
\end{center}
\end{frame}

\begin{frame}
\frametitle{Descripción del Modelo}
\framesubtitle{Última ejecución}
\begin{center}
\includegraphics[scale=0.4]{paso4}
\end{center}
\end{frame}

\begin{frame}
\frametitle{Gracias por la atención}
\includegraphics[scale=0.12]{gracias}
\end{frame}
\end{document}
