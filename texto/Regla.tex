\chapter{Regla de independencia y prueba no clausal de teoremas}

En el presente capítulo se expondrá el diseño de un método de prueba de teoremas basado en la consistencia o inconsistencia del conjunto de fórmulas de partida, así como de la negación de la fórmula que se quiere deducir. En definitiva se basa en el hecho de que, si $K$ es una base de conocimiento y $F$ una fórmula proposicional:

$$K\vDash F \;\text{ si y sólo si }\; K \cup \{ \neg F \} \text{ es inconsistente}$$

La idea principal será hallar la inconsistencia del conjunto de fórmulas mediante la saturación de dicho conjunto en la constante $\bot$. Para saturar, se usarán las ya mencionadas retracciones conservativas,  que se calcularán mediante lo que se denominará un \textit{operador de omisión de variables}. \\

Este operador eliminará una de las variables cada vez que se aplique, obteniendo a cada paso un conjunto equivalente de fórmulas en los que se usa una variable menos. Finalmente, como hay un número finito de variables en $K \cup \{ \neg F \} $, llegará un momento en el que no queden variables y que sólo queden las constantes $\top$ o $\bot$ (sólo una de ellas), habiendo así refutado o probado, respectivamente, el teorema.\\

Un cambio relevante en la estructura de este capítulo respecto a la seguida en el capítulo anterior es que se expondrán en primer lugar las bases teóricas que fundamentan el modelo lógico-algebraico de razonamiento; y, posteriormente, se implementará en Haskell dicho modelo. 

\section{Retracción conservativa mediante omisión de variables}
En esta sección se presenta cómo calcular retracciones conservativas usando los ya mencionados operadores \textit{de omisión (o de olvido)}. Dichos operadores son mapas del tipo:
$$\delta : Form(\mathcal{L}) \times Form(\mathcal{L}) \longrightarrow Form(\mathcal{L}) $$
donde $2^X$ representa al conjunto potencia de $X$. 

\defn Sea $\delta$ un operador: $\delta :Form(\mathcal{L}) \times Form(\mathcal{L}) \longrightarrow Form(\mathcal{L} \setminus \{ p \})$  se dice que es:

\begin{enumerate}
\item \textit{robusto} si $\{F,G\} \vDash \delta (F,G)$.
\item un \textit{operador de omisión} para la variable $p \in \mathcal{L}$ si:
$$\delta (F,G) \equiv [\{F,G\}, \mathcal{L} \setminus \{p\}]$$
\end{enumerate} 

Una caracterización muy útil de los operadores se puede deducir de la siguiente propiedad semántica: Si  $\delta$ es un operador de omisión, los modelos de $\delta (F,G)$ son precisamente las \textit{proyecciones} de los modelos de $\{ F,G \}$ (ver figura \ref{fig:proy}). 

\vspace{0.5cm}
\begin{figure}[h]
	\centering
		\includegraphics[scale=0.6]{imagenes/indemod.png}
	\caption{Interpretación semántica del operador de omisión (Lema de elevación)}
	\label{fig:proy}
\end{figure}
\vspace{0.5cm}

\lem \label{lem:lifting} (\textbf{Lema de elevación}) Sean $v :\mathcal{L} \setminus \{p\} \rightarrow \{ 0,1 \}$ una valoración o interpretación, $F, G \in Form(\mathcal{L})$ fórmulas y $\delta$ un operador de omisión de la variable $p$. Las siguientes condiciones son equivalentes:
\begin{enumerate}
\item $v \vDash \delta (F,G)$
\item Existe una valoración $\hat{v} : \mathcal{L} \rightarrow \{ 0,1 \}$ tal que $\hat{v} \vDash F \wedge G$ y $\hat{v} \upharpoonright_{\mathcal{L} \setminus \{ p \}} = v $
\end{enumerate}

\noindent \textbf{Prueba: } ($1 \Rightarrow 2$): Dada una valoración $v$, se considera la fórmula 
$$H_v = \bigwedge_{q \in \mathcal{L} \setminus \{ p \}} q^v$$
donde $q^v$ es $q$ si $v(q)=1$ y $\neg q$ en otro caso. Es claro que $v$ es la única valoración de $\mathcal{L} \setminus \{ p \}$ que es modelo de $H_v$. \\
Supongamos que existe $v  :\mathcal{L} \setminus \{p\} \rightarrow \{ 0,1 \}$ modelo de $\delta (F,G)$, pero que no se puede extender a un modelo de $F \wedge G$. Entonces la fórmula $$H_v \rightarrow \neg (F \wedge G)$$ es una tautología, en particular:
$$ \{ F,G \} \vDash H_v \rightarrow \neg (F \wedge G)$$
Como $ \{ F,G \} \vDash F \wedge G$, usando \textit{modus tollens} se tiene $\{ F,G \} \vDash \neg H_v$. Así que $\delta (F,G) \vDash \neg H_v$, por ser $\delta$ un retracción conservativa. Este hecho es una contradicción porque $v \vDash \delta (F,G) \wedge H_v$.\\
($2 \Rightarrow 1$): La extensión $\hat{v}$ verifica que:
$$\hat{v} \vDash F \wedge G \vDash [\{ F,G \} , \mathcal{L} \setminus \{ p \} ] \vDash \delta (F,G)$$
Como $\delta (F,G) \in Form(\mathcal{L} \setminus \{ p \})$, la valoración $v = \hat{v} \upharpoonright_{\mathcal{L} \setminus \{ p \}}$ también es modelo de $\delta (F,G)$. \hspace{14cm} $\square$ \\

En particular, el resultado es cierto para la propia retracción conservativa canónica $[K, \mathcal{L} \setminus \{ p \}]$, porque si consideramos la fórmula $\bigwedge K := \bigwedge_{F \in K} F$,
$$[K, \mathcal{L} \setminus \{ p \}] \equiv \delta_p (\bigwedge K , \bigwedge K)$$

Un caso interesante aparece cuando $\delta_p(F_1,F_2) \equiv \top$. En este caso, toda valoración parcial en $\mathcal{L} \setminus \{ p \}$ se puede extender a un modelo de $\{ F_1,F_2 \}$.\\

La siguiente caracterización será útil más adelante:
\cor Sea $\delta : Form(\mathcal{L}) \times Form(\mathcal{L}) \longrightarrow Form(\mathcal{L} \setminus \{ p \})$ un operador robusto. Las siguientes condiciones son equivalentes:
\begin{enumerate}
\item $\delta$ es un operador de omisión de la variable $p$.
\item Para cualesquiera $F,G \in Form(\mathcal{L})$ y $v \vDash \delta (F,G) $ valoración sobre $\mathcal{L} \setminus \{ p \}$, existe una extensión de $v$ modelo de $\{ F,G \}$.
\end{enumerate}

\noindent \textbf{Prueba:} ($1 \Rightarrow 2$): Cierta por el Lema de Elevación (Lema \ref{lem:lifting}).\\
($2 \Rightarrow 1$): Sean $F$ y $G$ dos fórmulas. Como $\delta$ es robusto, basta probar que:
$$\delta (F,G) \vDash [\{ F,G \}, \mathcal{L} \setminus \{ p \}]$$
Supongamos que no es cierto. En ese caso, existe una fórmula $H \in Form(\mathcal{L} \setminus \{ p \})$ tal que $[\{ F,G \}, \mathcal{L} \setminus \{ p \}] \vDash H$ (luego $H$ también es consecuencia lógica de $\{ F,G \}$), pero existe una valoración $v$ que satisface $v \vDash \delta (F,G) \wedge \neg H$. Por $(2)$, existe $\hat{v}$ extensión de $v$ que es modelo de $\{ F,G \}$. Por tanto, $\{ F,G \} \vDash \neg H$, lo que es una contradicción. \hspace{7.7cm} $\square$ \\

\cor Si $p \notin var(F)$, y $\delta_p$ es un operador de omisión de $p$, entonces
$$\delta_p(F,F) \equiv F \;\;\;\;\;\;\;\; \text{y} \;\;\;\;\;\;\;\; \delta_p (F,G) \equiv \{ F,\delta_p (G,G) \}$$

\noindent \label{cor:pnotinvar} \textbf{Prueba:} Si $p \notin var(F)$, entonces $\{ F \} \equiv [\{ F \} ,\mathcal{L} \setminus \{ p \}] \equiv \delta_p (F,F)$.\\
Por otro lado, $\delta_p (F,G) \equiv [\{ F,G \} , \mathcal{L} \setminus \{ p \}] \vDash \{ F,\delta_p (G,G) \}$. Para probar que en realidad se trata de una equivalencia se mostrara que tienen los mismos modelos.\\
Sea $v$ una valoración sobre $\mathcal{L} \setminus \{ p \}$ tal que $v \vDash \{ F,\delta_p (G,G) \}$. Entonces existe $\hat{v}$ (una extensión de $v$) tal que $\hat{v} \vDash G$. Como $\hat{v} \vDash F$ se tiene por el Lema de Elevación que $v \vDash \delta (F,G)$. \hspace{14cm} $\square$ 
 
\subsubsection{Retracciones conservativas inducidas por un operador de omisión} 
 
En el artículo \cite{Lang2003} J. Lang et al.  presentan un método de omisión de $X$ (un conjunto de variables de la fórmula $F$), denotado por $\texttt{forget}(F,X)$ y basado en la construcción de disyunciones de la siguiente forma:\\

\begin{tabular}{lll}
$\texttt{forget}(F, \emptyset)$ & $=$ & $F$\\
$\texttt{forget}(F, \{ x \})$ & $= $ & $F\{x/\top \} \cup F\{x/\bot \}$ \\
$\texttt{forget}(F, \{ x \} \cup Y)$ & $= $ & $\texttt{forget} (\texttt{forget}(F,Y),\{ x \})$
\end{tabular}

\vspace{0.5cm}

Notar que con este método el tamaño de $\texttt{forget}(F,Y)$ puede ser realmente grande. En el método que se expone en el trabajo se pretende simplificar la representación mediante el uso de operaciones algebraicas sobre proyecciones polinomiales.\\

Tal y como se ha descrito anteriormente, el operador de omisión de la variable $p$ actúa entre pares de fórmulas. A continuación, se extenderá la definición del operador de forma que se pueda aplicar a conjuntos de fórmulas o bases de conocimiento.

\defn Sea $\delta_p$ un operador de omisión de la variable $p$ y $K$ una base de conocimiento. Se define $\delta_p [\cdot ]$ como:\\

\begin{tabular}{l}
$\delta_p [\cdot ] : 2^{Form(\mathcal{L})} \rightarrow 2^{Form(\mathcal{L})}$ \\
$\delta_p [K] := \{ \delta_p (F,G) : F,G \in K \}$
\end{tabular}

\vspace{0.5cm}

Si se supone que se tiene un operador de omisión $\delta_p$ para cada $p\in \mathcal{L}$:

\defn Se llamará \textit{saturación} de la base de conocimiento $K$ al proceso de aplicar los operadores $\delta_p [\cdot ]$ (en algún orden) respecto a todas las variables proposicionales de $\mathcal{L}(K)$, denotando al resultado como $sat_{\delta}(K)$ (el cual será un subconjunto de $\{ \top , \bot \}$).\\

Posteriormente se verá que $sat_{\delta}(K)$ no depende del orden de aplicación de los operadores. Además, se probará que debido a que los operadores de omisión son robustos, si $K$ es consistente entonces necesariamente $sat_{\delta}(K)=\{ \top \}$.\\

A partir de los operadores de omisión resulta natural definir el siguiente cálculo lógico:

\defn Sea $K$ una base de conocimiento, $F\in Form(\mathcal{L})$ y $\{ \delta_p : p \in \mathcal{L}(K) \}$ una familia de operadores de omisión.
\begin{itemize}
\item[•] $A \vdash_{\delta}$-prueba en $K$ es una secuencia de fórmula $F_1, \dots ,F_n$ tal que para todo $i \leq n$, $F_i \in K$ ó existen $F_j , F_k (j,k < i)$ tal que $F_i = \delta_p (F_j , F_k)$ para algún $p \in \mathcal{L}$.
\item[•] $K \vdash_{\delta} F$ si existe una $\vdash_{\delta}$-prueba en $K$, $F_1, \dots ,F_n$, con $F_n = F$.
\item[•] Una $\vdash_{\delta}$-refutación es una $\vdash_{\delta}$-prueba de $\bot$.
\end{itemize}

La completitud (refutacional) del cálculo asociado a los operadores de omisión se enuncia como sigue:

\thm Sea $\{ \delta_p : p \in \mathcal{L} \}$ una familia de operadores de omisión. Entonces $\vdash_{\delta}$ es refutacionalmente completo, es decir, $K$ es inconsistente si y sólo si $K \vdash_{\delta} \bot$.\\

\noindent \textbf{Prueba:} La idea es saturar la base de conocimiento como en la Figura (\ref{fig:comple}). Si $sat_{\delta} (K) = \{ \top \}$, entonces, aplicando repetidas veces el lema de elevación, se puede extender la valoración vacía (la cual es modelo de $\{ \top \}$) a un modelo de $K$.\\
Si $\bot \in sat_{\delta} (K)$ entonces $K$ es inconsistente, porque $K \vDash sat_{\delta} (K)$ por robustez de los operadores de omisión. $\square$ %% La elección de una $\vdash_{\delta}$-refutación en particular es sencilla.

\vspace{0.5cm}
\begin{figure}[h]
	\centering
		\includegraphics[scale=0.44]{imagenes/comple.png}
	\caption{Decidir la consistencia usando operadores de omisión ($\partial_{p_i}$)}
	\label{fig:comple}
\end{figure}
\vspace{0.5cm}

\cor $\delta_p [K] \equiv [K, \mathcal{L} \setminus \{ p \}]$

\noindent \textbf{Prueba:} Por robustez del operador de omisión $\delta_p$ se tiene:
$$[K, \mathcal{L} \setminus \{ p \}] \vDash \delta_p [K]$$

Para probar la otra dirección, se considera $F\in [K, \mathcal{L} \setminus \{ p \}]$ y se supone que $\delta_p [K] \nvDash F$. Entonces $\delta_p [K] + \{ \neg F \}$ es consistente. En particular, si se satura se tiene: $sat_{\delta}(\delta_p [K] \cup \{ \neg F \}) = \{ \top \}$.\\
Como $p \notin var(\neg F)$ se puede usar el corolario \ref{cor:pnotinvar}, obteniendo que para todo $G\in K$:
$$\delta_g (\neg F, G) \equiv \{ \neg F, \delta_g  (G,G)  \} \;\;\;\;\;\;\;\;\;\;\;\; \text{y} \;\;\;\;\;\;\;\;\;\;\;\; \delta_p (\neg F , \neg F) \equiv \neg F$$
Por consiguiente,
$$\delta_p [K \cup \{ \neg F \}] \equiv \delta_p [K] \cup \{ \neg F \}$$
así que, aplicando saturación empezando por $p$:
$$sat_{\delta}(K \cup \{ \neg F \}) \equiv sat_{\delta}(\delta_p [K] \cup \{ \neg F \})$$
Lo que indica que $K \cup \{ \neg F \}$ es consistente, luego $K \nvDash F$, que es una contradicción. $\square$\\

Dado $Q \subseteq \mathcal{L}$ y un orden lineal $q_1 < \cdots < q_k$

\section{Derivadas Booleanas}

\section{Regla de independencia}

\section{Cálculo lógico}